
En el presente trabajo se realiza la implementación de una red neuronal sigmoidea con una capa oculta, usando los lenguajes C99 y Assembly x64. Debido a que dicho tipo de redes se basa fuertemente en operaciones vectoriales, encontramos en este proyecto una buena excusa para probar distintas optimizaciones usando SIMD.
\\
\\
En las próximas secciones se explicará el trasfondo teórico de nuestro programa con el cual luego daremos una breve explicación de los detalles de la implementación que realizamos, llegando al final a los tiempos obtenidos donde intentaremos verificar si SIMD es realmente una buena alternativa para bajar los tiempos de computo de la red neuronal.\\
\\
El resultado final será un programa perfectamente funcional y capaz de identificar dígitos numericos con una muy buena precisión y con una interfaz cómoda para su fácil utilización.
