
En esta sección se explican los detalles implementativos de nuestro clasificador.

\subsection{Datos de Entrada}

El programa tiene el set de datos de MNIST ubicado en la carpeta $mnist$ en un archivo comprimido, este es convertido a archivos de texto que contienen un pixel por línea con un script de python el cual aloja los archivos en la carpeta $data$. Esto se hace para facilitar la posterior lectura de estos datos por nuestro programa.

Una vez hecho esto se puede utilizar program.py para compilar y ejecutar nuestro código de c que entrena una red neuronal para reconocer estos caracteres. Tanto en la carpeta $float$ como en la carpeta $double$ se encontrará el código que implementa dicho programa y las correspondientes herramientas de compilación. Para compilar los archivos fuente manualmente basta con utilizar el \texttt{Makefile}, este crea los archivos \texttt{asm\_version} y \texttt{c\_version} que ejecutan la red neuronal.

\subsection{Secciones Principales}

En las carpetas $double$ y $float$ hay versiones equivalentes del clasificador para estos dos tipos de datos.
\\
\\
Dentro de las carpetas hay 3 archivos con código fuente, $helpers$, $tensorOps$ y $nn$.
\\
\\
Los archivos $helpers$ implementan varias funciones útiles, entre otras cosas las que utilizamos para leer los TXTs donde estan los datos de entrenamiento, también algunas funciones matriciales básicas como transpocisión e impresión.
\\
\\
Luego esta $tensorOps$ en el cual pusimos los métodos que implementamos tanto en $C$ como en $ASSEMBLER$ estas son operaciones matriciales y vectoriales como el producto matricial y la suma vectorial.
\\
\\
Por último esta $nn$ que es el archivo que utiliza a los demás para implementar toda la lógica de la red neuronal y donde se encuentra el main de nuestro programa.

\subsection{Métodos implementados Assembler}

A continuación damos una breve explicación de los métodos que implementamos en assembler para intentar mejorar la performance se describen a continuación. Todos ellos poseen la propiedad de realizar cierta operación aritmética repetidas veces para distintos valores haciendo propicio el intento de paralelización con SIMD.

\subsubsection{cost\_derivative}

Este método es el gradiente del error cuadrático medio lo cual se reduce simplemente al computo de la resta entre vectores.
\\
\\
\begin{lstlisting}
void cost_derivative(double* res_vec, double* target_mat, uint cant_imgs, double* output) {
  for (int i = 0; i < 10; i++) {
    for (uint j = 0; j < cant_imgs; j++){
        output[i * cant_imgs + j] = res_vec[i * cant_imgs + j] - target_mat[i * cant_imgs + j];
    }
  }
}
\end{lstlisting}

\subsubsection{vector\_sum}

Como indica su nombre esta método computa la suma vectoria.
\\
\\
\begin{lstlisting}[frame=single]
void vector_sum(double* vector1, double* vector2, uint n, double* output){
// |vector1| == |vector2| == n
  for (int i = 0; i < n; i++) {
    output[i] = vector1[i] + vector2[i];
  }
}
\end{lstlisting}


\subsubsection{update\_weight}

COMPLETAR
\\
\\
\begin{lstlisting}[frame=single]
void update_weight(double* w, double* nw, uint w_size, double c){
  for(uint i = 0; i < w_size; i++){
    w[i] -= c * nw[i];
  }
}
\end{lstlisting}

\subsubsection{hadamard\_product}

El producto de Hadamard consiste en la multiplicación componente a componente entre dos matrices.
\\
\\
\begin{lstlisting}[frame=single]
void hadamard_product(double* matrix1, double* matrix2, uint n, uint m, double* output){
/* matrix1 and matrix2 are nxm*/
  for(uint i = 0; i < n; i++){
    for(uint j = 0; j < m; j++){
      output[i * m + j] = matrix1[i * m + j] * matrix2[i * m + j];
    }
  }
}
\end{lstlisting}

\subsubsection{matrix\_prod}

Este es el clásico producto matricial donde $matrix1$ es de $n \times m$, $matrix2$ es de $m \times l$ y la variable de salida, $output$, de dimensiones $n \times l$.
\\
\\
\begin{lstlisting}[frame=single,xleftmargin=1cm]
void matrix_prod(double* matrix1, double* matrix2, uint n, uint m, uint l, double* output){
// matrix1 is nxm
// matrix2 is mxl
// output is nxl
  for(uint i = 0; i < n; i++) {
    for(uint j = 0; j < l; j++){
      output[i * l + j] = 0;
      for(uint k = 0; k < m; k++){
        output[i * l + j] += matrix1[i * m + k] * matrix2[k * l + j];
      }
    }
  }
}
\end{lstlisting}


\subsection{Tests}

Tanto la versión del programa que recibe Floats como la que recibe Doubles tiene un conjunto de tests en las funciones que implementamos tanto en ASSEMBLER como C, esto fue para corroborar el buen funcionamiento de lo dessarrollado y ayudarnos a debuguear.
\\
\\
Dichos tests se encuentran en la carpeta $test$ y se compilan con el \texttt{Makefile} el cual crea el ejecutable $test$ que corre los mismos.