\subsection{Porcentaje de aciertos de las redes}

La red neuronal tiene un porcentaje de acierto del 95\% en todas sus versiones. A pesar de que el tipo float tiene una menor capacidad de representación numérica que su contraparte el tipo double su precisión parece ser suficiente para realizar los cálculos necesarios y hacer que la red neuronal llegue a las mismas conclusiones que la red que utiliza double.


\subsection{Tiempos}

Aquí se puede observar un cuadro de resultados comparando nuestras implementaciones de ASSEMBLER contra las de C compiladas con gcc -03.

\begin{center}
    \resizebox{\textwidth}{!}{
        \begin{tabular}{| l | c | c | c | c | c |}
                \hline
    Versión & cost\_derivative & mat\_plus\_vec & update\_weight & hadamard\_product & matrix\_prod \\
                \hline
    Double C & 0.019694 & 0.323289 & 0.298969 & 0.966852 & 0.841608 \\
    Double ASM & 0.019881 & 0.338240 & 0.758107 & 0.341586 & 0.498035 \\
    Float C & 0.019256 & 0.171635 & 0.158144 & 0.922496 & 0.711830 \\
    Float ASM  & 0.018904 & 0.170835 & 0.381136 & 0.167613 & 0.490826 \\
                \hline
			
        \end{tabular}}
\end{center}
\captionof{table}{ Cuadro de resultados obtenidos en cada método. }
